\documentclass[a4paper,10pt]{article}
\usepackage[utf8]{inputenc}
\usepackage[T2A]{fontenc}
\usepackage[russian]{babel}
\usepackage{geometry}
\usepackage{enumitem}
\usepackage{titlesec}
\usepackage{hyperref}
\geometry{margin=1.5cm}

\titleformat{\section}{\large\bfseries}{}{0em}{}
\renewcommand{\labelitemi}{--}

\begin{document}

\begin{center}
    {\LARGE \textbf{Кривощеков Александр Александрович}}\\
    \small C++ Developer (Системное программирование | Графика | Backend)\\
    Новосибирск | Email: \href{mailto:alex.krivoschecov@gmail.com}{alex.krivoschecov@gmail.com}  
Телефон: \href{tel:+79134178475}{+7 913 417 84 75}  
\end{center}

\section*{Резюме}
C++ разработчик с опытом в системном программировании, компьютерной графике и backend-разработке.  
Реализовал высокопроизводительные сервисы и игровые движки с использованием C++17/20/23, OpenGL/Vulkan, многопоточности и низкоуровневых оптимизаций.  
Опыт командной работы и руководства проектами (до 5 человек).  
Цель — разработка масштабируемых, оптимизированных систем.

\section*{Опыт работы}

\textbf{Turing Pi} | Удалённо — C++ Developer  
\textit{Июль 2024 — Апрель 2025 (контракт)}  
\begin{itemize}[noitemsep]
    \item Разработал микросервис на C++20 (Drogon), снизив задержки сообщений до $<$50 мс при 10k+ соединениях.
    \item Реализовал WebSocket API с низкими задержками и устойчивостью к обрывам соединений.
    \item Разработал AI-агентов для чат-системы (управление диалогом, контекстом).
    \item Перевёл проект с GNU Make на CMake, улучшив структуру сборки и сократив время компиляции на 25\%.
\end{itemize}

\section*{Проекты}

\textbf{Minecraft Clone (OpenGL 4.6)}  
\begin{itemize}[noitemsep]  
    \item Procedural generation на шуме Перлина, chunk-based мир.  
    \item Реализован pipeline: камера, освещение, текстуры, интерфейс.  
    \item 300 FPS @ 1440p (GTX 1050ti).  
    \item Стек: GLFW, GLEW, GLM, SFML.  
    \item Код: \url{https://github.com/cpp20120/OldGithub/tree/main/MineGL}  
    (GitHub: MineGL)  
\end{itemize}  

\textbf{Game Engine (C++23)}  
\begin{itemize}[noitemsep]
    \item Математическая библиотека с \texttt{mdspan}, TMP и Concepts для type-safe API.
    \item Архитектура модульного движка; планируется Vulkan-рендерер, ECS, скриптовая система.
    \item \url{https://github.com/cpp20120/Engine}{Github: Engine}.
\end{itemize}

\textbf{Webserver (C++20, io\_uring)}  
\begin{itemize}[noitemsep]
    \item \url{https://github.com/cpp20120/OldGithub/tree/main/WebServer} HTTP сервер на корутинах с io\_uring; асинхронная обработка запросов.
\end{itemize}

\textbf{ThreadPool (C++20, lock-free DAG runtime)}  
\begin{itemize}[noitemsep]
    \item Мини-рантайм для параллельных задач с DAG-ориентированным планировщиком.
    \item Собственный work-stealing пул (Chase–Lev) + центральные lock-free MPMC очереди (Michael–Scott) с Hazard Pointers/QSBR.
    \item Поддержка аффинити, приоритетов, back-pressure, ограничений конкуренции и явного управления переполнением (Block/Drop/Fail).
    \item Высокоуровневый API в стиле TBB: \texttt{submit}, \texttt{then}, \texttt{when\_all}, \texttt{parallel\_for}.
    \item Документация с архитектурными диаграммами и полным Doxygen-комментированием.
    \item Специализированный \texttt{for\_each\_ws} для RA-диапазонов (lazy range split + help-first stealing). 
    \item В моих условиях на 1e6 элементов дал \textasciitilde{}2{.}8× быстрее \texttt{oneTBB::parallel\_for} (RA-кейс). Общий \texttt{parallel\_for} и микротаски специально не оптимизировались.
    \texttt{Бенчмарки:}
    \item \url{https://github.com/cpp20120/ThreadPool}
\end{itemize}

\section*{Вклад в Open Source}
\begin{itemize}[noitemsep]
    \item Эксперимент по переносу hwloc (Hardware Locality) с autotools на CMake.
    \item Разработка библиотеки метапрограммирования (\url{https://github.com/cpp20120/MetaUtils}{MetaUtils}).
\end{itemize}

\section*{Технические навыки}
\begin{itemize}[noitemsep]
    \item \textbf{Языки и стандарты:} C++17/20/23, TMP, Concepts, Coroutines
    \item \textbf{Графика:} OpenGL 4.6, Vulkan, GLSL, SPIR-V
    \item \textbf{Backend:} Drogon, Boost.Beast/Asio, REST, WebSocket
    \item \textbf{Метапрограммирование:} CRTP, SFINAE, constexpr вычисления, шаблонные интерпретаторы
    
    \item \textbf{Многопоточность:} std::thread, TBB, folly
    \item \textbf{Low-latency оптимизации:} io\_uring,  zero-copy, lock-free (Michael–Scott queue)
    \item \textbf{CI/CD и тесты:} GitHub Actions, CTest, clang-format/tidy, GTest
    \item \textbf{Инструменты:} CMake, Ninja, Git, vcpkg, RenderDoc, Nsight
    \item \textbf{БД:} PostgreSQL (оптимизация запросов)
    \item \textbf{OS:} Linux (low-level), Windows
\end{itemize}

\section*{Образование}
\textbf{Сибирский государственный университет телекоммуникаций и информатики}  
Бакалавриат, «Информатика и вычислительная техника»  
\textit{2023 — 2027 (ожидаемый выпуск)}

\section*{Дополнительно}
\begin{itemize}[noitemsep]
    \item Английский — B1 (чтение документации, участие в код-ревью)
    \item Увлечения: компьютерная графика, алгоритмы
\end{itemize}

\section*{Преподавательская деятельность}
\begin{itemize}[noitemsep]
    \item Провёл цикл неформальных занятий по Git для одногруппников (охват 20+ человек)
    \item Темы: от базовых команд до продвинутых практик (интерактивный rebase, submodules)
    \item Результат: 100\% группы начали использовать Git в учебных проектах
\end{itemize}

\end{document}
