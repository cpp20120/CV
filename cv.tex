\documentclass[a4paper,10pt]{article}
\usepackage[utf8]{inputenc}
\usepackage[T2A]{fontenc}
\usepackage[russian]{babel}
\usepackage{geometry}
\usepackage{enumitem}
\usepackage{titlesec}
\usepackage{hyperref}
\geometry{margin=1.5cm}

\titleformat{\section}{\large\bfseries}{}{0em}{}
\renewcommand{\labelitemi}{--}

\begin{document}

\begin{center}
    {\LARGE \textbf{Кривощеков Александр Александрович}}\\
    \small C++ Developer (Системное программирование | Графика | Backend)\\
    Новосибирск | \href{mailto:alex.krivoschecov@gmail.com}{alex.krivoschecov@gmail.com} | +7 913 417 84 75\\
    \href{https://github.com/cpp20120}{github.com/cpp20120}
\end{center}

\section*{Профессиональное резюме}
C++ разработчик с опытом в системном программировании, компьютерной графике и backend-разработке.  
Реализовал высокопроизводительные сервисы и игровые движки с использованием C++17/20/23, OpenGL/Vulkan, многопоточности и низкоуровневых оптимизаций.  
Опыт командной работы и руководства проектами (до 5 человек).  
Цель — разработка масштабируемых, оптимизированных систем.

\section*{Опыт работы}

\textbf{Turing Pi} | Удалённо — C++ Developer  
\textit{Июль 2024 — Апрель 2025 (контракт)}  
\begin{itemize}[noitemsep]
    \item Разработал микросервис на C++20 (Drogon), снизив задержки сообщений до $<$50 мс при 10k+ соединениях.
    \item Реализовал WebSocket API с низкими задержками и устойчивостью к обрывам соединений.
    \item Разработал AI-агентов для чат-системы (управление диалогом, контекстом).
    \item Перевёл проект с GNU Make на CMake, улучшив структуру сборки и сократив время компиляции на 25\%.
\end{itemize}

\section*{Проекты}

\textbf{Minecraft Clone (OpenGL 4.6)}  
\begin{itemize}[noitemsep]
    \item Procedural generation на шуме Перлина, chunk-based мир.
    \item Реализован pipeline: камера, освещение, текстуры, интерфейс.
    \item 300 FPS @ 1440p (GTX 1050ti).
    \item Стек: GLFW, GLEW, GLM, SFML.
    \item \href{https://github.com/cpp20120/OldGithub/tree/main/MineGL}{Исходный код}.
\end{itemize}

\textbf{Game Engine (C++23)}  
\begin{itemize}[noitemsep]
    \item Математическая библиотека с \texttt{mdspan}, TMP и Concepts для type-safe API.
    \item Архитектура модульного движка; планируется Vulkan-рендерер, ECS, скриптовая система.
    \item \href{https://github.com/cpp20120/Engine}{Исходный код}.
\end{itemize}

\textbf{Webserver (C++20, io\_uring)}  
\begin{itemize}[noitemsep]
    \item \href{https://github.com/cpp20120/OldGithub/tree/main/WebServer} HTTP сервер на корутинах с io\_uring; асинхронная обработка запросов.
\end{itemize}

\section*{Вклад в Open Source}
\begin{itemize}[noitemsep]
    \item Эксперимент по переносу hwloc (Hardware Locality) с autotools на CMake.
    \item Разработка библиотеки метапрограммирования (\href{https://github.com/cpp20120/MetaUtils}{MetaUtils}).
\end{itemize}

\section*{Технические навыки}
\begin{itemize}[noitemsep]
    \item \textbf{Языки и стандарты:} C++17/20/23, TMP, Concepts, Coroutines
    \item \textbf{Графика:} OpenGL 4.6, Vulkan, GLSL, SPIR-V
    \item \textbf{Backend:} Drogon, Boost.Beast/Asio, REST, WebSocket
    \item \textbf{Многопоточность:} std::thread, TBB, folly, lock-free структуры
    \item \textbf{CI/CD и тесты:} GitHub Actions, CTest, clang-format/tidy, GTest
    \item \textbf{Инструменты:} CMake, Ninja, Git, vcpkg, RenderDoc, Nsight
    \item \textbf{БД:} PostgreSQL (оптимизация запросов)
    \item \textbf{OS:} Linux (low-level), Windows
\end{itemize}

\section*{Образование}
\textbf{Сибирский государственный университет телекоммуникаций и информатики}  
Бакалавриат, «Информатика и вычислительная техника»  
\textit{2023 — 2027 (ожидаемый выпуск)}

\section*{Дополнительно}
\begin{itemize}[noitemsep]
    \item Английский — B1 (чтение документации, участие в код-ревью)
    \item Увлечения: компьютерная графика, алгоритмы
\end{itemize}

\end{document}
