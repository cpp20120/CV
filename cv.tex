\documentclass[a4paper,10pt]{article}
\usepackage[utf8]{inputenc}
\usepackage[T2A]{fontenc}
\usepackage[russian]{babel}
\usepackage{geometry}
\usepackage{enumitem}
\usepackage{titlesec}
\usepackage{hyperref}
\usepackage{graphicx}
\usepackage[export]{adjustbox} 
\geometry{margin=1.5cm}

\titleformat{\section}{\large\bfseries}{}{0em}{}
\titleformat{\subsection}{\normalsize\bfseries}{}{0em}{}

\renewcommand{\labelitemi}{--}

\begin{document}
	
	\begin{center}
		\textbf{\LARGE Кривощеков Александр Александрович} \\
		\small C++ Developer (System Programming | Graphics | Backend) \\
		Новосибирск \\
		\href{mailto:alex.krivoschecov@gmail.com}{alex.krivoschecov@gmail.com} | +7 913 417 84 75 \\
		\href{https://github.com/cpp20120}{github.com/cpp20120} 
	\end{center}
	
	\section*{О себе:}
	C++ разработчик с опытом в системном программировании, компьютерной графике и backend-разработке. Специализируюсь на высокопроизводительных решениях с использованием C++17/20. Имею опыт руководства командами в университетских проектах, где организовывал разработку и координировал задачи. Стремлюсь к созданию оптимизированных и масштабируемых систем.
	
	\section*{Опыт работы}
	\textbf{Turing Pi} | Удалённо — C++ Developer \\
	\textit{Июль 2024 — Апрель 2025 по контракту}
	\begin{itemize}[noitemsep]
		\item Разработал микросервис для чат-системы на C++20 (Drogon Framework).
		\item Реализовал WebSocket API с поддержкой низких задержек.
		\item Разработал AI-агентов для чат-системы, включая управление диалогом и контекстом..
		\item Инициировал и реализовал переход проекта с GNU Make на CMake, улучшив структуру сборки.
	\end{itemize}
	
	\section*{Участие в хакатонах}
	\begin{itemize}[noitemsep]
		\item \textbf{VTB MORE.Tech (10.2023)} — Разработка сервиса для поиска ближайшего отделения банка. Реализовал A* алгоритм для навигации.
	\end{itemize}
	
	\section*{Ключевые проекты}
	
	\subsection*{Minecraft Clone (OpenGL 4.6)} 
	\begin{itemize}[noitemsep]
		\item Реализация procedural generation с использованием шума Перлина
		\item Реализован базовый pipeline: камера, освещение, текстуры, интерфейс.
		\item Chunk based мир
		\item GLFW GLEW GLM OpenGL + SFML для интерфейса
		\item 300 fps 1440p на GTX 1050ti
		\item Исходный код: \href{https://github.com/cpp20120/OldGithub/tree/main/MineGL}{github.com/cpp20120/OldGithub/tree/main/MineGL}
	\end{itemize}
	
	\subsection*{Game Engine (Разработка собственного игрового движка )}
	\begin{itemize}[noitemsep]
		\item Собственная математическая библиотека на C++23 (с использованием mdspan).
		\item Концепты и шаблонное метапрограммирование (TMP) для type-safe API и метапрограммирования.
		\item Проектирование архитектуры для масштабируемых модулей.
		\item Планы: Vulkan-рендерер, ECS, скриптовая система.
		\item Исходный код: \href{https://github.com/cpp20120/Engine}{github.com/cpp20120/Engine}
	\end{itemize}

    \subsection*{Webserver}
	\begin{itemize}[noitemsep]
        \item Реализовал http веб сервер на C++ 20 с использованием корутин и iouring
	\end{itemize}
	
	\subsection*{Курсовые проекты}
	\begin{itemize}[noitemsep]
		\item Разработка GUI-приложений на Qt Widgets
		\item Понимание принципов сигналов/слотов, модели представления данных
		\item Опыт работы с QMainWindow, стандартными виджетами и layout-ами
        \item Личный курсовой проект
	\end{itemize}
    \begin{itemize}
        \item В командных курсовых проектах руководил разработкой задания. Делегировал задачи участникам команды. Строил архитектуру проекта и координировал команду из 5 человек, обеспечив сдачу проекта в срок.
    \end{itemize}
	
	\section*{Вклад в OpenSource}
	\begin{itemize}[noitemsep]
		\item Экспериментировал с переходом hwloc (Hardware Clock lib) с autotools на modern CMake.  
		\item Столкнулся со сложностями из-за tightly coupled bash-логики и отказался от полного порта.  
		\item Приобрёл опыт работы с legacy build-системами.  
		\item Релизовал библиотеку метапрограммирования с большим количеством функционала. 
        
        Исходный код: \href{https://github.com/cpp20120/MetaUtils}{github.com/cpp20120/MetaUtils}
	\end{itemize}
	
	\section*{Технические навыки}
	\begin{tabular}{@{}ll@{}}
		\textbf{C++} & Modern C++17/20, TMP, Concepts, Coroutines \\
		\textbf{CI/CD} & GitHub Actions, CTest, clang-format/tidy \\
		\textbf{Unit-тесты} & GTest \\
		\textbf{Графика } & OpenGL 4.6, Vulkan, GLSL, SPIR-V \\
		\textbf{Backend} & Drogon, Poco, Boost.Beast/Asio, REST/WebSocket \\
		\textbf{Базы данных} & PostgreSQL (оптимизация запросов) \\
		\textbf{Многопоточность} & TBB, folly, std::thread, lock-free структуры \\
		\textbf{Инструменты} & CMake, Ninja, Git, vcpkg, RenderDoc, Nsight \\
		\textbf{OS} & Linux (системное программирование), Windows \\
	\end{tabular}
	
	\section*{Образование}
	\textbf{Сибирский государственный университет телекоммуникаций и информатики} \\
	Бакалавриат по направлению "Информатика и вычислительная техника" \\
	\textit{2023 — 2027 (ожидаемый выпуск)} \\
	
	\section*{Дополнительная информация}
	\begin{itemize}[noitemsep]
		\item Английский язык — Intermediate (чтение технической документации)
		\item Увлечения: компьютерная графика, алгоритмы
		\item Готов рассматривать вакансии, связанные с Qt-разработкой (как entry-level)
	\end{itemize}
\end{document}
